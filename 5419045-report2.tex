\documentclass[dvipdfmx]{jsarticle}
\usepackage[T1]{fontenc}
\usepackage[dvipdfmx]{hyperref}
\usepackage{lmodern}
\usepackage{latexsym}
\usepackage{amsfonts}
\usepackage{amssymb}
\usepackage{mathtools}
\usepackage{nccmath}
\usepackage{amsthm}
\usepackage{multirow}
\usepackage{graphicx}
\usepackage{wrapfig}
\usepackage{here}
\usepackage{float}
\usepackage{ascmac}
\usepackage{url}

\title{CGによるスタイル表現の比較と実験}
\author{文理学部情報科学科\\5419045 高林 秀}
\date{\today}

\begin{document}

\maketitle

\begin{abstract}
  本稿では、今年度マルチメディア情報処理の第2回目課題研究として、「blender」及び「Google Colablatry」における3DCGの描画をもちいて、配布されたデータ「Lit-Sphire」と[StyleBlit]を使用しCG画像のスタイル表現の動画化を実験し、それぞれのデータにおける結果を比較・考察するものである。結果は、、、、で、、、であった。
\end{abstract}

\section{目的}
\section{関連技術調査}
\section{実験方法}
  \subsection{実験準備}
    \subsubsection{実験環境}
    今回の実験は仮想マシン上でR言語を起動し行った。下記に実験時の環境を示す。
    \begin{itemize}
      \item ホストOS:Window10 Home Ver.20H2
      \item 仮想OS:Ubuntu 20.04.2 LTS
      \item CPU:Intel(R)Core(TM)i7-9700K @ 3.6GHz
      \item GPU:Nvidia Geforce RTX2070 OC @ 8GB
      \item ホストRAM:16GB
      \item 仮想RAM:4GB
    \end{itemize}
\section{実験結果}
\section{考察}
\section{まとめ}


\section{巻末付録}
\begin{itemize}
  \item 使用したソースコード、画像等へのリンク:\url{https://drive.google.com/drive/folders/1ciY7XHNFSsUBXfiAC9xqw2D2_LYDKVSv?usp=sharing}
\end{itemize}
\begin{thebibliography}{99}

\end{thebibliography}

\end{document}
