\documentclass[dvipdfmx]{jsarticle}
\usepackage[T1]{fontenc}
\usepackage[dvipdfmx]{hyperref}
\usepackage{lmodern}
\usepackage{latexsym}
\usepackage{amsfonts}
\usepackage{amssymb}
\usepackage{mathtools}
\usepackage{nccmath}
\usepackage{amsthm}
\usepackage{multirow}
\usepackage{graphicx}
\usepackage{wrapfig}
\usepackage{here}
\usepackage{float}
\usepackage{ascmac}
\usepackage{url}

\title{CGによるスタイル表現の比較と実験}
\author{文理学部情報科学科\\5419045 高林 秀}
\date{\today}

\begin{document}

\maketitle

\begin{abstract}
  本稿では、今年度マルチメディア情報処理の第2回目課題研究として、「blender」及び「Google Colablatry」における3DCGの描画をもちいて、配布されたデータ「Lit-Sphire」と[StyleBlit]を使用しCG画像のスタイル表現の動画化を実験し、それぞれのデータにおける結果を比較・考察するものである。結果は、、、、で、、、であった。
\end{abstract}


\end{document}
